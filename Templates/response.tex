\documentclass[12pt]{article}
\usepackage[margin=1in]{geometry}
\usepackage{natbib}
\usepackage{enumerate,amsmath,amssymb}
\usepackage{rotating}
\usepackage[colorlinks, linkcolor = blue, citecolor = blue]{hyperref}
\parskip=3mm
\newenvironment{comment}%
{\begin{quotation}\small\it\noindent\ignorespaces%
}{\end{quotation}}


%\usepackage{color} % include if you want to have colored text
%\newcommand{\blue}{\color{blue}}  


\begin{document}
\begin{center} \Large \bf
  Response to Reviewer
\end{center}

\subsection*{Summary}

Begin by thanking the reviewer for carefully reading your work. When 
you submit the response-to-reviewers document with the first draft, 
the reviewer is your peer reviewer. When you submit the 
response-to-reviewers document with the final paper, the reviewer is your 
instructor.  

Refer to Note 7 on the HuskyCT STAT4916W website for more details, guidelines, 
and tips.

The manuscript has been revised accordingly with the following major changes:
\begin{enumerate}
\item I made this broad-level change;
\item I also made this broad-level change;
\item Finally, I also made this broad-level change.  
\item ~[Note: the items here are just for summarizing the changes; the 
specific responses to each reviewer comment are to be provided in the 
Point-by-point responses below.]
\end{enumerate}

The point-by-point responses are as follows.

\subsection*{Point-by-point responses}

\textbf{Major Comments}

\begin{comment}
	This is where you insert the reviewer's first major comment. Copy their 
	concern verbatim here. It will be indented and appear in italics.
\end{comment}


This is where you respond to the reviewer's first major comment.


\begin{comment}
	This is where you insert the reviewer's second major comment, again  
	taken verbatim from the reviewer report.  
\end{comment}


This is where you respond to the reviewer's second major comment. 


\begin{comment}
Page 2, line 18: It is stated that ``Another author proposes a solution
that disentangles the parameters of interest using limited data'' but then later
in line 27 you appear to contradict yourself by stating ``the limited data is not 
fit for our purpose.''  Can you please clarify?
\end{comment}

Thank you for identifying this issue. The apparent contradiction is due to the 
lack of clarity in my original statements.  The approach in the literature 
allows parameters of the model to be estimated, but the properties of the 
underlying data generating process cannot be determined.  Therefore we conclude
that the limited data is not fit for our purpose.  We have change the first
quoted sentence to be more clear in the revised manuscript: 
``Another author proposes a solution that disentangles the parameters of 
interest using limited data, but the properties of the underlying data 
generating process cannot be determined.''


\begin{comment}
Pages 3-4, Section 2: Given that a key aspect of the paper is data 
visualization, one way to improve the paper would be to include more graphical
displays.  More specifically, images that are more exciting and tell a stronger
story about the data would be useful in Section 2. 
\end{comment}

This is a great suggestion. I have now added density plots and side-by-side 
boxplots to show the distribution of house sale prices (on the log-scale) 
for the 10 largest cities in Connecticut in 2020 (see Figures 2 and 3 in the 
revised manuscript).

\begin{comment}
	This is where you insert the reviewer's next major comment, again  
	taken verbatim from the reviewer report.  
\end{comment}

This is where you respond to the reviewer's next major comment. 

\bigskip

Continue in this fashion until you have addressed \textit{all} of the reviewer's 
concerns, both major and minor.

\textbf{Minor Comments}

\begin{comment}
\item Page 3, lines 18-20: What is meant by ``As we collected data, we 
periodically discarded entries that did not have location data and contain
one of our attributes of interest.''  Does ``not''  modify both clauses?
\end{comment}

No, ``not'' does not modify both clauses.  Just is now clarified in the revised
manuscript as: ``Observations without location data or other attributes of 
interest were discard from the analysis.'' 

\begin{comment}
\item Page 7, line 5: Change ``was'' to ``were'' in ``The data was collected 
$\ldots$''.
\end{comment}

Thank you, revised as such.  

\bigskip

If references need to be cited anywhere above, the complete references should be 
included here at the end (using, e.g., refs.bib with mcap.sty in the 
commented-out text below.)

%\bibliography{refs}
%\bibliographystyle{mcap} 

\end{document}